\documentclass[10pt,twoside]{article}

\usepackage{amsmath}
\usepackage{graphicx,epsfig}
\usepackage{graphicx}
\usepackage{amsmath,amssymb,amsbsy,bm}
%\usepackage[framed]{mcode}

\newlength{\toppush}
\setlength{\toppush}{2\headheight}
\addtolength{\toppush}{\headsep}

\renewcommand{\bottomfraction}{0.95}

\newcommand{\htitle}[3]{\begin{center}
\vspace*{-\toppush}
{\large MASSACHUSETTS INSTITUTE OF TECHNOLOGY}\\
{\small Department of Electrical Engineering and Computer Science}\\
\vspace*{1ex}{\Large #2}\end{center}
\noindent
\newline\parbox{6.5in}
{Fall 2013\hfill Issued : #1 \newline
 Problem Set 9 \hfill Due : #3\newline
%\profs \hfill %Handout #1\vspace*{-.5ex}\newline
%\mbox{}\hrulefill\mbox{}
}}

\newcommand{\mcO}{\mathcal{O}}
\newcommand{\handout}[3]{\thispagestyle{empty}
\pagestyle{myheadings}\htitle{#1}{#2}{#3}}

\setlength{\oddsidemargin}{0pt}
\setlength{\evensidemargin}{0pt}
\setlength{\textwidth}{6.5in}
\setlength{\topmargin}{0in}
\setlength{\textheight}{8.5in}


\newcommand{\pp}[2]{\frac{\partial #1}{\partial #2}}%
\newcommand{\ppp}[2]{\frac{\partial^2 #1}{\partial #2^2}}%
\newcommand{\dd}[2]{\frac{d #1}{d #2}}%
\newcommand{\ddd}[2]{\frac{d^2 #1}{d #2^2}}%
\newcommand{\matend}{\end{array}\right]}
\newcommand{\matc}{\left[\begin{array}{c}}
\newcommand{\matcc}{\left[\begin{array}{cc}}
\newcommand{\bb}{\mathbf{b}}
\newcommand{\bx}{\mathbf{x}}
\newcommand{\bA}{\mathbf{A}}
\newcommand{\DD}[2]{\frac{D #1}{D #2}}%
\newcommand{\Uvec}{\mathbf{U}}
\newcommand{\uvec}{\mathbf{u}}
\newcommand{\tauvec}{\bm{\tau}}
\newcommand{\omegavec}{\bm{\omega}}


\renewcommand{\Re}{\mathrm{Re}}


\begin{document}


\handout{Dec 10, 2013}{6.301 Solid State Circuits}{Not due}
\setlength{\parindent}{0pt}

\newcommand{\solution}{
 \medskip
 {\bf Solution:}
}

\hrulefill

\flushleft
\vspace{3ex}
\hangindent=1cm {\bf Problem 1:} \quad In this problem, a transistor is controlled by supplying a base current drive as seen in Figure 1.
Analyze the dynamics of the transistor and sketch $q_F, q_S, i_C$, and $i_B$ versus time. Each sketch should
clearly indicate important slopes, final values, time constants, etc. Read this whole problem before
starting to solve the first part.
\begin{enumerate}
	\item[1.]
Assume that the transistor remains in the forward active region. Determine the time constants and
final values etc, and sketch the curves.
	\item[2.]
Since $\beta_Fi_B > i_{C_{SAT}}$, the device will not remain in the forward active region for all time. Indicate
on your graphs the point at which saturation occurs. Find $q_{BO}$, the final values (in saturation) for $i_C$ and $q_S$ and evaluate the time constant $\tau_S$.  Continue the sketches.  How long does it take to traverse the active region?
	\item[3.]
Now consider turning off the device, $i_B=0$.  Assume that athe transistor remains saturated, that is $i_C=i_{C_{SAT}}$ for all time.  Determine the final values for $q_S$ and sketch the curves.
	\item[4.]
Obviously, the transistor does not remain saturated but enters the forward active region when $q_S$ equals zero. Determine the storage delay time, that is, the time during which the device remains saturated even though $i_B=0$. Determine the time spent crossing the active region. Sketch the curves.
	\item[5.] Compare turn-on times and turn-off times: explain the difference.
	\item[6.] It is observed that the storage delay time decreases if the input pulse duration is reduced.  Explain.
\end{enumerate}
\vspace{2ex}
{\bf Problem 2:} Charge Control
\begin{enumerate}
	\item[1.] Sketch $q_F, q_S, i_C$, and $i_B$ versus time for the circuit shown in Figure 2.
	\item[2.] Now assume that you are free to choose the capacitor value. what value should be chosen so
that final conditions for both the turn on and the turn off transient are established as quickly as
possible?
\end{enumerate}
\clearpage
\end{document}
