\documentclass[11pt, a4paper]{article}

\usepackage{float}
\usepackage{amsmath}
\usepackage{graphicx,epsfig}
\usepackage{enumerate}
\usepackage[margin=.8in]{geometry}
\providecommand{\e}[1]{\ensuremath{\times 10^{#1}}}

\title{MEng Thesis Proposal}
\author{Josh Gordonson}

\begin{document}
\maketitle

\section{Motivation}
The breadboard has been a staple substrate for electronic construction over the last century.
At the dawn of a growing interest in amatuer radio, resourceful tinkerers used planks of wood to secure and ruggedize their electrical handiwork.
%Breadboarding was a construction technique that enabled reconfigurable experiments or permanent electrical machines 
Conductive nodes, such as nails or copper rails, were driven into the non-conductive boards, providing anchors and contact points that were electrically isolated from the rest of the circuit.
Components were soldered or wire-wrapped to the nodes, and sometimes secured by non-energized nails or screws.
This construction technique provided a lot of artistic freedom in circuit construction, but was relatively time consuming and required relatively heavy hand tools, such as a hammer or drill.

The solderless breadboard is the cannonical tool given to students taking introductory courses in the field.
Rather than driving nodes into arbitrary locations, component leads are inserted into contact points that allow rapid semi-rigid construction of circuits with no other tools.
The layout of a solderless breadboard is designed to be compatible with a plethora of powerful integrated circuits, enabling complex designs.
Modern solderless breadboards have come a long way from their namesake wooden ancestors, but there is still room for improvement. 

The shortcomings of solderless breadboards lie entirely within the art of breadboarding.
The intent of breadboarding is to physically realize a circuit.
Often, this involves designing or using a reference schematic to guide construction, but circuit improvisation is not uncommon.
Inserting components and jumper wires into contact points is straightforward, but poor contacts, broken wires (inside insulation), and mis-inserting leads can plague designers for hours on end or even destroy components.
A meticulous breadboarder can successfully realize a circuit without error, but for the uninitiated it is difficult to justify the additional time and care required to plan and build.
Breadboarding is a skill that is learned over time, but small errors can lead to excessive frustration and turn students off to the field.
To satisfy the requirements of the Masters in Engineering program, I propose a solution to some of the issues with the solderless breadboard.

\section{Overview}


\begin{thebibliography}{1}
\end{thebibliography}
\end{document}

