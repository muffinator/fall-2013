%\def\DEBUG{1}
\ifdefined\DEBUG
\documentclass[11pt,twoside]{mitthesis}
\usepackage{tikz}
\usepackage{circuitikz}
\usepackage{amsmath}
\usepackage{fancyvrb}
\usepackage{float}
\newcommand{\ohm}{$\Omega$ }
\begin{document}
\fi

\chapter{Software Control}
In this chapter, software control of the hardware system is covered.
The software that controls the hardware system is very similar to the software simulation in Chapter 3.  
A control routine parses first circuit 'experiment' requests and configures the hardware to follow out the experiment.
The routine then analyzes the returned measurement data.
Unlike in the simulation, the measurement data is returned in the time domain rather than the frequency domain.
The software transforms the returned data to the frequency domain to calculate the magnitude of the returned waveforms.
With magnitude data, the same network sensing algorithm from Chapter 2  is used to reconstruct the network.
\section{Control Routine}

"Just a teaspoon of sugar makes the medicine go down" - Mary Poppins

The control routine is quite self-explanatory.

\begin{Verbatim}[fontsize=\footnotesize]
def medicine(drive,chan,freq=1000,adc=0,uglist=[]):
    nog=0	#a number representing nodes not to ground
    for x in uglist:
        nog+=(1<<x)
    ser.write(['g',0xff&(~((1<<chan)|(1<<drive)|nog))])	
		#ground all of the nodes whose bits are set to 1
    ser.write(['w',drive+1])
    	#connect the voltage source to node 'drive'
    freq = int(freq)
    f=[freq&0xff,(freq&0xff00)>>8,(freq&0xff0000)>>16]
    	#send three bytes representing the desired frequency
    ser.write(['m',chan])
    	#set the voltage multiplexer to the channel of interest, chan
    ser.flushInput()
    	#clear everything already in the input buffer
    ser.write((['s']+f+[adc]))
    	#tell the microcontroller to begin sampling
    	# adc value of 0 is for voltage and 1 for current
    a = ser.read(2000)
    b = [5*((ord(a[2*x+1])<<8)+ord(a[2*x]))/1024. for x in range(len(a)/2)]
    b = b[2:]
    	# the ADC data is 10-bits per sample, but the serial channel only supports
    	# sending 8-bits at a time, so it has to be reconstructed.
    ft=fft(b)
    aind=argmax(abs(ft[1:]))+1
    	#calculate the magnitude of the most prominent sinusoid and return it
    amp = abs(ft[aind])/500
    return amp
\end{Verbatim}

\subsection{FFT}
The normalized FFT is used to extract the amplitude data out of the measured sinusoidal voltages and currents.
Since the measured voltages and currents should be close to purely sinusoidal, the magnitude of the FFT of those voltage and currents should be close to a delta function at the frequency of interest.
The magnitude peak value of the normalized FFT is the amplitude of the measured sinusoidal voltages and currents.

\section{Network Sensing Algorithm}
 The implementation of the network sensing algorithm in the hardware control software is identical to that in the hardware simulator, mentioned in Chapter 3.
%Unfortunately, inductors tend to generate noisy impedance measurements, which heavily defeats the finite difference method of finding the slope of the impedance vs frequency plots.

\ifdefined\DEBUG
\end{document}
\fi