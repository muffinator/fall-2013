\def\DEBUG{1}
\ifdefined\DEBUG
\documentclass[11pt,twoside]{mitthesis}
\usepackage{tikz}
\usepackage{circuitikz}
\usepackage{amsmath}
\usepackage{float}
\newcommand{\ohm}{$\Omega$ }
\begin{document}
\fi

\chapter{Results}
In this chapter, results for the simulation and hardware implementations of the RLC network identifying system are presented and analyzed.
The shortcomings of the systems are addressed and potential improvements are discussed as future work.

\section{Simulation Performance}
The performance of the RLC network identifying system simulator is analyzed by its output precision and cases in which it fails.
The output precision of the simulator is quite good and can be controlled almost arbitrarily.
The precision depends almost entirely on the thresholds defined during the finite-differencing step, as looser thresholds allow bad data to get averaged into the good data.
The cases in which the simulator fails are always cases where a relatively high resistance is in parallel with an LC network.
In these cases, the simulator reports that there is no resistor in parallel with the LC network.
The failure mode lies within finite difference stencil thresholding, where the simulator never finds a region where the slope of the impedance vs frequency plot is zero, and in turn the simulator decides there is no resistor.
Designing a better element identifying algorithm would make excellent future work.

%This could be remedied by either a patch to the algorithm or another algorithm entirely.
%Instead of using a slope of zero on the \texttt{|Z| vs. f} plot to indicate parallel resistance, the maximum recorded impedance can be taken as the parallel resistance whenever the simulator finds an LC circuit.
%In the case of an LC circuit where there is no resistance, 
\section{Hardware Performance}
\subsection{Block Performance}
\subsubsection{}
\subsection{Capacitive Coupling}
\subsection{Limits of Operation}
Theoretical limits of system performance based on block performance

\section{System Performance}
\subsection{Speed of Operation}
\subsection{Accuracy}
\subsection{Dynamic Range}
\subsection{Odd Behavior}
Capacitors

\section{Improvements}
\subsection{L, C, D}
\subsection{VGA for ADC's, DAC, and Differential Amplifier}
\subsection{Variable Sense Resistor}
\subsection{Better Switches}
\subsection{Three Terminal Devices}
Try Transistors
\subsection{Better Schematic Display}
\subsection{Faster Algorithm}

\ifdefined\DEBUG
\end{document}
\fi