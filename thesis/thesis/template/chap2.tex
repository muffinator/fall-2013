%\def\DEBUG{1}
\ifdefined\DEBUG
\documentclass[11pt,twoside]{mitthesis}
\usepackage{tikz}
\usepackage{circuitikz}
\usepackage{amsmath}
\usepackage{float}
\newcommand{\ohm}{$\Omega$ }
\begin{document}
\fi

\chapter{Theory}
In this chapter, a Network Sensing Algorithm that can reverse-engineer an arbitrary n-node RLC network given access to every node is described.
First, it is shown how basic circuit theory is used to analyze and reverse engineer n-node resistive circuits.
These methods are then generalized to reverse-engineer n-node impedance networks down to the individual impedance branch circuits, constructed from resistors, capacitors, and inductors.
The impedance networks are then further decomposed into their constituent element by method of frequency-domain analysis.
Finally, the original network is reconstructed.

\section{RLC Elements}

The design of a Network Sensing Algorithm requires an understanding of the networks to be analyzed.
In this thesis, the networks of interest are composed of three circuit elements: resistors, capacitors, and inductors.
The constitutive relations for these electrical circuit elements relate the voltage across an element to the current passing through it as a function of time.
The constitutive relation for resistors, inductors, and capacitors are as follows:

Resistive element behavior is characterized by a linear relationship between current and voltage.
That is, $v=iR$

\begin{figure}[H]
  \begin{center}
    \begin{circuitikz}[american]
		\draw (3,0)
		to[sV,v=$v$,i=$i$] (3,2)
		to[short](5,2)
		to[R=$\displaystyle {R=\frac{v}{i}}$] (5,0)
		to[short](3,0); 
        \end{circuitikz}
   \caption{Constitutive Relations for Resistance}
  \end{center}
\end{figure}

Inductive element behavior is characterized by a linear relationship between the time derivative of current and voltage.
That is, $v=L\frac{di}{dt}$

\begin{figure}[H]
  \begin{center}
    \begin{circuitikz}[american]
		\draw (3,0)
		to[sV,v=$v$,i=$i$] (3,2)
		to[short](5,2)
		to[L=$\displaystyle {L=\frac{v}{\frac{di}{dt}}}$] (5,0)
		to[short](3,0); 
        \end{circuitikz}
   \caption{Constitutive Relations for Inductance}
  \end{center}
\end{figure}

Capacitive element behavior is characterized by a linear relationship between the time derivative of voltage and current.
That is, $i=C\frac{dv}{dt}$

\begin{figure}[H]
  \begin{center}
    \begin{circuitikz}[american]
		\draw (3,0)
		to[sV,v=$v$,i=$i$] (3,2)
		to[short](5,2)
		to[C=$\displaystyle {C=\frac{i}{\frac{dv}{dt}}}$] (5,0)
		to[short](3,0); 
        \end{circuitikz}
   \caption{Constitutive Relations for Capacitance}
  \end{center}
\end{figure}

When combined to form larger networks these constitutive relations hold, but two other laws of conservation are useful for further analysis.

\section{Network Analysis}

\subsubsection{Kirchoff's Current Law}
Kirchoff's current law comes from the idea of conservation of charge.
That is, charge is neither created nor destroyed.
The amount of charge moving along a conductor away from a circuit node is equal to the amount of charge moving along a conductor into that same circuit node.
More succinctly, The sum of the currents going into a node are equal to the sum of currents coming out of that node.

\subsubsection{Kirchoff's Voltage Law}
Kirchoff's voltage law can be derived from conservation of energy.
KVL states that the total amount of energy put into a circuit is equal to the amount of energy lost in that circuit.
Since voltage is a measure of the amount of energy per unit charge, the voltage across a circuit branch is indicative of the amount of energy dissipated as work or provided by an electromotive force.
Following the last two statements, the sum of all of the voltages around a loop in a circuit is zero.

The following sections will use KVL. KCL, and the constitutive relations to reverse-engineer an n-node network.

\ifdefined\DEBUG
\end{document}
\fi
