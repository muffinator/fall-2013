%\documentclass[11pt,twoside]{mitthesis}
%\usepackage{tikz}
%\usepackage{circuitikz}
%\usepackage{amsmath}
%\usepackage{float}
%\newcommand{\ohm}{$\Omega$ }
%\begin{document}

\chapter{Theory}
The design of a Network Sensing Algorithm requires an understanding of the networks to be analyzed.
In this thesis, the networks of interest are composed of three circuit elements: resistors, capacitors, and inductors.
The constitutive relations for these electrical circuit elements relate the voltage across an element to the current passing through it as a function of time.
In this application, it is useful to reinterpret the constitutive relations as a function of frequency by taking the Fourier Transform.


\section{RLC Elements}

The constitutive equations that define current and voltage relations in the elements of interest are as follows:

Resistive element behavior is characterized by a linear relationship between current and voltage.
That is, $v=iR$

\begin{figure}[H]
  \begin{center}
    \begin{circuitikz}[american]
		\draw (3,0)
		to[sV,v=$v$,i=$i$] (3,2)
		to[short](5,2)
		to[R=$\displaystyle {R=\frac{v}{i}}$] (5,0)
		to[short](3,0); 
        \end{circuitikz}
   \caption{Constitutive Relations for Resistance}
  \end{center}
\end{figure}

Inductive element behavior is characterized by a linear relationship between the time derivative of current and voltage.
That is, $v=L\frac{di}{dt}$

\begin{figure}[H]
  \begin{center}
    \begin{circuitikz}[american]
		\draw (3,0)
		to[sV,v=$v$,i=$i$] (3,2)
		to[short](5,2)
		to[L=$\displaystyle {L=\frac{v}{\frac{di}{dt}}}$] (5,0)
		to[short](3,0); 
        \end{circuitikz}
   \caption{Constitutive Relations for Inductance}
  \end{center}
\end{figure}

Capacitive element behavior is characterized by a linear relationship between the time derivative of voltage and current.
That is, $i=C\frac{dv}{dt}$

\begin{figure}[H]
  \begin{center}
    \begin{circuitikz}[american]
		\draw (3,0)
		to[sV,v=$v$,i=$i$] (3,2)
		to[short](5,2)
		to[C=$\displaystyle {C=\frac{i}{\frac{dv}{dt}}}$] (5,0)
		to[short](3,0); 
        \end{circuitikz}
   \caption{Constitutive Relations for Capacitance}
  \end{center}
\end{figure}

\section{Frequency Domain Perspective}

Taking the 
Laplace transform of:\\
$v=iR, v=L\frac{di}{dt},$ and $i=C\frac{dv}{dt}$\\
$V=IR, V=LsI, I=CsV$

Impedance:\\
$Z_R=R, Z_L=Ls, Z_C=\frac{1}{Cs}$

\section{Network Analysis}

KVL
KCL

%\end{document}
