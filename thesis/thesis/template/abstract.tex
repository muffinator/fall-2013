% $Log: abstract.tex,v $
% Revision 1.1  93/05/14  14:56:25  starflt
% Initial revision
% 
% Revision 1.1  90/05/04  10:41:01  lwvanels
% Initial revision
% 
%
%% The text of your abstract and nothing else (other than comments) goes here.
%% It will be single-spaced and the rest of the text that is supposed to go on
%% the abstract page will be generated by the abstractpage environment.  This
%% file should be \input (not \include 'd) from cover.tex.


%A system for identifying networks of circuit elements was designed and constructed in the pursuit of better educational electronic prototyping tools.
%This document contains an analysis of a circuit network identification system, with a focus on determining RLC networks using impedance sensing techniques.  
%A design of both hardware and software implementations of the circuit network identification system are presented, and the results of the implementations are analyzed.
%Also presented is a method for surface-mounting breadboard finger springs to printed circuit boards.





The hands-on side of electrical engineering is still taught using solderless breadboards.  To lower the learning curve and improve the utility of solderless breadboards, I have designed and implemented a prototype that draws schematic diagrams of passive circuits that are built on a breadboard.  The system reverse engineers circuits by means of a network sensing algorithm, which iteratively grounds and excites nodes with voltage sources, and subsequently measures the resulting currents and voltages in the network.  Both a software simulation and a hardware implementation were built to test the network sensing algorithm.  The sofware system is capable of reverse-engineering arbitrarily sized RLC networks with some caveats regarding high-q parallel RLC networks.  The hardware system is able to accurately detect resistive and capacitive networks with eight nodes, though current hardware limitations significantly reduce the precision of measurment.  The performance of the hardware system was analyzed and solutions to many of the measurement issues were found.  A technique for surface-mount soldering breadboards to PCBs is presented in this thesis.

