%% This is an example first chapter.  You should put chapter/appendix that you
%% write into a separate file, and add a line \include{yourfilename} to
%% main.tex, where `yourfilename.tex' is the name of the chapter/appendix file.
%% You can process specific files by typing their names in at the 
%% \files=
%% prompt when you run the file main.tex through LaTeX.
\chapter{Introduction}

\section{History of Breadboards}

The breadboard has been a staple substrate for electronic construction over the last century.
At the dawn of a growing interest in amateur radio, resourceful tinkerers used planks of wood to secure and ruggedize their electrical handiwork.
%%Breadboarding was a construction technique that enabled reconfigurable experiments or permanent electrical machines
%%---------------------------------BREADBOARD FIGURE-------------------------------
Conductive nodes, such as nails or copper rails, were driven into the non-conductive boards, providing anchors and contact points that were electrically isolated from the rest of the circuit.
Components were soldered or wire-wrapped to the nodes, and sometimes secured by non-energized nails or screws.
This construction technique provided a lot of artistic freedom in circuit construction, but was time consuming and required relatively heavy hand tools such as a hammer or drill.
From a performance perspective, sensitive high-frequency circuits were infeasible due to a lack of a ground plane and long wires between components.


\section{Modern Breadboards}

The solderless breadboard is the canonical tool given to students taking introductory courses in the field.
Rather than driving nodes into arbitrary locations, component leads are inserted into contact points arranged on a grid that allow rapid semi-rigid construction of circuits with no other tools.
%%-------------------------------BREADBOARD FIGURE-----------------------------
The layout of a solderless breadboard is designed to be compatible with a plethora of powerful integrated circuits - enabling complex electronic designs - but is still limited to low-frequency operation due to the parasitic capacitance between adjacent breadboard rails.
Modern solderless breadboards have come a long way from their namesake wooden ancestors, but there is still room for improvement.

The intent of breadboarding is to physically realize a circuit.
Often, this involves designing or using a reference schematic to guide construction, but circuit improvisation is not uncommon.  
A meticulous breadboarder can successfully realize a circuit without error by correctly placing components and jumper wires - taking care not to introduce undesired 'parasitic' components.  
However, for the uninitiated it is difficult to justify the additional time and care required to plan and build.
Inserting components and jumper wires into contact points is straightforward, but poor contacts, broken wires (inside insulation), and mis-inserting leads can plague designers for hours on end and potentially destroy components.
%The shortcomings of solderless breadboards lie entirely within the art of breadboarding.
Breadboarding is a skill that is learned over time, but small errors can lead to excessive frustration and turn students off to the field.
I propose a solution to some of the issues with the solderless breadboard.

A confident linkage between the the electrical and mechanical domains is required to construct a circuit.
On larger scales, the mechanical structures (eg. contacts, wires, components) that create electrical circuits are visible in plain sight.
It is simple and reliable, then, to determine the circuit representation of a mechanical structure that has no hidden connections.
Breadboards often obscure connections due to their construction.
The regular nature of a breadboard, a grid of contact points arranged in rails, makes it difficult to keep track of which rail is connected to what circuit node.
When combined with the opaque nature of most components and the poor reliability of breadboard contacts, visually verifying complex circuits on a breadboard becomes infeasible.
Many of the inherent problems with breadboards stem from this open-loop nature of breadboarding, where visual cues are not enough to determine the electrical circuit from the mechanical structure.
A symptom of open-loop construction techniques is a mismatch between the mental model and the physical realization of the system at hand.
I seek to close this loop by designing and implementing a circuit-sensing breadboard.

\section{Proposed Solution}

A circuit-sensing breadboard is able to reverse-engineer circuits built on it by applying test voltages and currents and measuring the results.  
As a proof-of-concept, the proposed circuit-sensing breadboard completed for this thesis is able to reverse engineer circuits composed of resistors, inductors, and capacitors built on a small section of the breadboard and display the circuits as a schematic in the browser.

The circuit-sensing breadboard is composed of a hardware system that interfaces various pieces of test equipment with eight rails on a breadboard.
The system is composed of a pcb-mounted breadboard, a voltage source, current meter, and voltage meter multiplexing board, a microcontroller development board, firmware to control ADC sampling, AC voltage source testing, and streaming data to a computer for analysis and display.
The hardware is controlled by a circuit sensing algorithm, which determines the equivalent circuit elements between any two nodes on the breadboard.
A software simulation of the hardware system verifies operation of the network sensing algorithm and enables testing for networks larger than eight nodes.
Once the circuit sensing algorithm reconstructs the entire circuit it generates a schematic diagram of the circuit, closing the loop on breadboard construction.

%Since software is faster to prototype with, the theoretical circuit topology work was first implemented in software.
%A circuit simulator test-bench was written to provide the circuit-sensing algorithm in development with data.
%This test-bench allows the circuit-sensing algorithm to probe, stimulate, and ground every node of a randomly generated circuit, as if the circuit were built on a breadboard attached to the hardware test bench mentioned above.
%The software test-bench also had the ability to print out a circuit network solution to a schematic display.


\section{Implementation to Date}


Made the software test bench, an 8-rail hardware test bench, and got the network sensing algorithm working for resistors with resistance between 100 and 5K $\Omega$.




