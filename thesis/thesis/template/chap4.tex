%\documentclass[11pt,twoside]{mitthesis}
%\usepackage{tikz}
%\usepackage{circuitikz}
%\usepackage{amsmath}
%\begin{document}

\chapter{Simulation}

A simulation was built to test the Network Sensing Algorithm before embarking on hardware design.
In addition to the rapid prototyping cycle, the implementation of NSA and automated schematic drawing were directly used later on.
The simulation generates random networks of resistors, inductors, and capacitors, and proceeds to analyze and reconstruct the network using NSA.

\section{NgSpice and Netlists}

The open source software package NgSpice was used to simulate the networks and test circuits applied to the network.
NGspice operates on text files called netlists, where each component in the network is specified on a single line.
An example netlist is shown below.

\begin{figure}[h]
  \begin{center}
    \begin{circuitikz}[american]
    %\ctikzset{label/align = straight}
		\def\offset{0}
		\draw (-1+\offset,-2)
		to[short,o-] (-1+\offset,-2.548)
		node[sground] {} (-1+\offset,-2.548);
		\draw (-.75+\offset,-2)
		to[open,v^>=$V_A$] (-.75+\offset,-.1)
		to[open](-1+\offset,0)
		to[short,o-](0+\offset,0);
		\draw (\offset,0)
		node[label={above:$A$}] {}
		to[R, l=$R_{AB}$] (3+\offset,0)
		node[label={above:$B$}] {}
		to[L, l=$L_{BC}$] (1.5+\offset,-2.548)
		node[label={right:$C$}] {}
		to[C, l=$C_{AC}$] (0+\offset,0);
		\draw (1.5+\offset,-2.548)
		to[short]
		node[sground] {} (1.5+\offset,-2.548);
		\draw (4+\offset,-2.548)
		node[sground] {}
		to[V,v_>=$V_t$] (4+\offset,0)
		to[short] (3+\offset,0)
		;
		\fill (\offset,0) circle (1mm);
		\fill (1.5+\offset,-2.584) circle (1mm);
		\fill (3+\offset,0) circle (1mm);
    \end{circuitikz}
   \caption{Five node network}
  \end{center}
\end{figure}

\begin{verbatim}
fiveNodeNetlist
Vt	0 B 1
Rab A B 10k
Cac A C 1e-6
Lbc B C 1e-3
Vcg 0 C 0
.control
	op
	print(v(A))
	.endc
.end
\end{verbatim}

The first letter of each element line designates the element to simulate: \\
V start stop value$\rightarrow$ Voltage Source between start and stop nodes, value Volts [V]\\
R start stop value$\rightarrow$ Resistor between start and stop nodes, value Ohms [$\Omega$]\\
L start stop value$\rightarrow$ Inductor between start and stop nodes , value Henries [H]\\
C start stop value$\rightarrow$ Capacitor between start and stop nodes, value Farads [F]\\

Node 0 is always designated as ground, and all simulations require a ground node.

The control commands are as follows:\\
op $\rightarrow$ Operating Point Simulation
print() $\rightarrow$ Print the relevant data passed as an argument
v(N) $\rightarrow$ The voltage at node N
i(E) $\rightarrow$ The current through element E

\section{Methods}
The NSA simulator was written in python and uses the methods below.

\subsection{Generate Random Netlist}
\texttt{writeRandomNet(netlist,num,elements):}\\
Generates \texttt{num}-node random graph with no self-linking nodes (symmetric matrix with zeros on the diagonal) for each \texttt{[elements]} type (R,L,C).
Subsequently assigns random values between two realistic limits for each element and writes the network to netlist \texttt{netlist}.
1-1k ohms, 10nF-10uF, 100uH-100mH.

When writing the capacitive and inductive elements, care must be taken to prevent a DC operating point simulation from failing.
The infinite resistance across a capacitor and zero resistance across an inductor are responsible for DC operating point simulation failure, and can be fixed by including a small resistor in series with inductors and a large resistor in parallel with capacitors. 

\begin{tabular}{ c c  p{3cm} }
\texttt{L0 1 2 1mH} & $\rightarrow$ & \texttt{L0 1 tl0 1mH Rl0 tl0 2 1e-5} \\
\end{tabular}
\begin{tabular}{ |c  c  p{3cm} }
\texttt{C0 1 2 1e-6F} & $\rightarrow$ & \texttt{C0 1 2 1e-6F Rc0 1 2 1e8} \\
\end{tabular}

\subsection{Inserting Voltage Sources, Grounds and Probes}
\texttt{insertProbe2(target,nodes,groundNodes,probes,source='DC'):}\\
Inserts a 1V voltage source from ground to each node in \texttt{[nodes]}, grounds each node in \texttt{[groundNodes]}, and adds a voltage print statement for each node in \texttt{[probes]}.
By default, the voltage sources are written as DC sources, but if 'AC' is passed into the last argument the sources are written as AC sources and the AC control statement is added.\\
\texttt{AC dec 1 1 100000}\\
Which runs a small-signal AC simulation and returns the amplitudes of the resulting voltage and current waveforms, one sample point per decade from 1Hz to 100kHz.

\subsection{Run Simulation}
\texttt{def runSim(target,results,source='DC'):}\\
Makes a system call to NgSpice in batch mode with netlist \texttt{target} and outputs the result to text file \texttt{results}.
The last argument indicates how to parse the resulting data, as NgSpice returns DC data in the following format:\\
\texttt{
No. of Data Rows : 1\\
i(v) = -1.18295e-01\\
v(5) = 2.532846e-07}\\
and AC data is returned in this format:\\
\texttt{
No. of Data Rows : 6\\
                                   mynetlist\\
                                   AC Analysis  Sun Aug 30 18:35:07  2015\\
--------------------------------------------------------------------------------\\
Index   frequency       i(v)                            \\
--------------------------------------------------------------------------------\\
0	1.000000e+00	-1.72598e+00,	3.978810e+02	\\
1	1.000000e+01	-1.58689e-01,	3.978827e+01	\\
2	1.000000e+02	-1.43015e-01,	3.974287e+00	\\
3	1.000000e+03	-1.42859e-01,	3.520201e-01	\\
4	1.000000e+04	-1.42857e-01,	-4.18884e-01	\\
5	1.000000e+05	-1.42857e-01,	-4.58275e+00	\\
\\
                                   mynetlist\\
                                   AC Analysis  Sun Aug 30 18:35:07  2015\\
--------------------------------------------------------------------------------\\
Index   frequency       v(3)                            \\
--------------------------------------------------------------------------------\\
0	1.000000e+00	1.000000e+00,	0.000000e+00	\\
1	1.000000e+01	1.000000e+00,	0.000000e+00	\\
2	1.000000e+02	1.000000e+00,	0.000000e+00	\\
3	1.000000e+03	1.000000e+00,	0.000000e+00	\\
4	1.000000e+04	1.000000e+00,	0.000000e+00	\\
5	1.000000e+05	1.000000e+00,	0.000000e+00\\
}

\subsection{Print Matrix}
\texttt{def printMatrix(m):}
Prints matrix \texttt{m} in nice command-line output.

\section{Executing NSA}
\subsection{Calculate $Z_{n{||}}(f)$}
$Z_{n{||}}(f)$ is found by grounding all nodes except for node n and adding a voltage source to that node, then taking the ratio of the amplitudes of the resulting current into the node of interest and the voltage source.
\subsection{Calculate $V_n(f)$}
$V_n(f)$ is found by grounding all nodes except for nodes n and m, adding a voltage source to node m, and measuring the amplitude of the voltage at node n.
\subsection{Calculate $Z_{nm}(f)$}
$Z_{nm}(f)$ is calculated by the ratio of $Z_{n{||}}(f)$ to $V_n(f)$ scaled by $V_t$.
In the case of this simulation, $V_t$ is one.
\subsection{Finite Difference}

\subsection{Element Identification}

\subsection{Network Reconstruction}

\section{Output to JSON}
Blah Blah

\section{D3?}
???? maybe

%\end{document}