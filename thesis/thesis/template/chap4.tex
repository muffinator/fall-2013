\documentclass[11pt,twoside]{mitthesis}
\usepackage{tikz}
\usepackage{circuitikz}
\usepackage{amsmath}
\begin{document}
\chapter{Simulation}

A simulation was built to test the Network Sensing Algorithm before embarking on hardware design.
In addition to the rapid prototyping cycle, the implementation of NSA and automated schematic drawing were directly used later on.
The simulation generates random networks of resistors, inductors, and capacitors, and proceeds to analyze and reconstruct the network using NSA.

\section{NgSpice and Netlists}

The open source software package NgSpice was used to simulate the networks and test circuits applied to the network.
NGspice operates on text files called netlists, where each component in the network is specified on a single line.
An example netlist is shown below.

\begin{figure}[h]
  \begin{center}
    \begin{circuitikz}[american]
    %\ctikzset{label/align = straight}
		\def\offset{0}
		\draw (-1+\offset,-2)
		to[short,o-] (-1+\offset,-2.548)
		node[sground] {} (-1+\offset,-2.548);
		\draw (-.75+\offset,-2)
		to[open,v^>=$V_A$] (-.75+\offset,-.1)
		to[open](-1+\offset,0)
		to[short,o-](0+\offset,0);
		\draw (\offset,0)
		node[label={above:$A$}] {}
		to[R, l=$R_{AB}$] (3+\offset,0)
		node[label={above:$B$}] {}
		to[L, l=$L_{BC}$] (1.5+\offset,-2.548)
		node[label={right:$C$}] {}
		to[C, l=$C_{AC}$] (0+\offset,0);
		\draw (1.5+\offset,-2.548)
		to[short]
		node[sground] {} (1.5+\offset,-2.548);
		\draw (4+\offset,-2.548)
		node[sground] {}
		to[V,v_>=$V_t$] (4+\offset,0)
		to[short] (3+\offset,0)
		;
		\fill (\offset,0) circle (1mm);
		\fill (1.5+\offset,-2.584) circle (1mm);
		\fill (3+\offset,0) circle (1mm);
    \end{circuitikz}
   \caption{Five node network}
  \end{center}
\end{figure}

\begin{verbatim}
fiveNodeNetlist
Vt	0 B 1
Rab A B 10k
Cac A C 1e-6
Lbc B C 1e-3
Vcg 0 C 0
.control
	op
	print(v(A))
	.endc
.end
\end{verbatim}

The first letter of each element line designates the element to simulate: \\
V start stop value$\rightarrow$ Voltage Source between start and stop nodes, value Volts [V]\\
R start stop value$\rightarrow$ Resistor between start and stop nodes, value Ohms [$\Omega$]\\
L start stop value$\rightarrow$ Inductor between start and stop nodes , value Henries [H]\\
C start stop value$\rightarrow$ Capacitor between start and stop nodes, value Farads [F]\\

Node 0 is always designated as ground, and all simulations require a ground node.


\section{Methods}
\subsection{Generate Random Netlist}
Generates a random graph with no self-linking nodes (symmetric matrix with zeros on the diagonal) for each element type (R,L,C).
Then assign random values between two realistic limits to each edge in each graph.
10-10k ohms, 1nF-10uF, 1uH-10mH.

When writing the capacitive netlist and the inductive netlist, care must be taken to prevent a DC operating point simulation from failing.
The infinite resistance across a capacitor and zero resistance across an inductor are responsible for DC operating point simulation failure, and can be fixed by including a small resistor in series with inductors and a large resistor in parallel with capacitors. 

\begin{tabular}{ c c  p{3cm} }
\texttt{L0 1 2 1mH} & $\rightarrow$ & \texttt{L0 1 tl0 1mH Rl0 tl0 2 1e-5} \\
\end{tabular}
\begin{tabular}{ |c  c  p{3cm} }
\texttt{C0 1 2 1e-6F} & $\rightarrow$ & \texttt{C0 1 2 1e-6F Rc0 1 2 1e8} \\
\end{tabular}

\subsection{Inserting Voltage Sources}

Voltage sources are inserted by adding a voltage source line 
\subsection{Inserting Voltage Probes}
\subsection{Inserting Grounds}

\section{Executing NSA}
Wouldn't we all?
\subsection{Calculate $Z_{||}(f)$}
\subsection{Calculate $V_n(f)$}
\subsection{Calculate $Z_{nm}(f)$}
\subsection{Finite Difference}
\subsection{Element Identification}
\subsection{Network Reconstruction}

\section{Output to JSON}
Blah Blah

\section{D3?}
???? maybe

\end{document}