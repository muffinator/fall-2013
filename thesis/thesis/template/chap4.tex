%\def\DEBUG{1}
\ifdefined\DEBUG
\documentclass[11pt,twoside]{mitthesis}
\usepackage{tikz}
\usepackage{circuitikz}
\usepackage{amsmath}
\usepackage{fancyvrb}
\usepackage{float}
\newcommand{\ohm}{$\Omega$ }
\begin{document}
\fi


\chapter{Simulation}
In this chapter, a simulation of the circuit sensing breadboard is described.
A simulated circuit sensing breadboard is useful for rapid iteration of hardware topologies and validation of network sensing algorithms.
The simulator is also not limited by the hardware complexity, so simulating circuits with more than eight nodes and elements other than resistors, capacitors, and inductors is possible.
The simulator generates random networks of resistors, inductors, and capacitors, and proceeds to analyze, reconstruct, and draw a schematic of the network using the network sensing algorithm.

\section{NgSpice and Netlists}

The open source software package NgSpice was used to simulate the networks and test circuits applied to the network.
NGspice operates on text files called netlists, where each component in the network is specified on a single line.
An example netlist is shown below.

\begin{figure}[h]
  \begin{center}
    \begin{circuitikz}[american]
    %\ctikzset{label/align = straight}
		\def\offset{0}
		\draw (-1+\offset,-2)
		to[short,o-] (-1+\offset,-2.548)
		node[sground] {} (-1+\offset,-2.548);
		\draw (-.75+\offset,-2)
		to[open,v^>=$V_A$] (-.75+\offset,-.1)
		to[open](-1+\offset,0)
		to[short,o-](0+\offset,0);
		\draw (\offset,0)
		node[label={above:$A$}] {}
		to[R, l=$R_{AB}$] (3+\offset,0)
		node[label={above:$B$}] {}
		to[L, l=$L_{BC}$] (1.5+\offset,-2.548)
		node[label={right:$C$}] {}
		to[C, l=$C_{AC}$] (0+\offset,0);
		\draw (1.5+\offset,-2.548)
		to[short]
		node[sground] {} (1.5+\offset,-2.548);
		\draw (4+\offset,-2.548)
		node[sground] {}
		to[V,v_>=$V_t$] (4+\offset,0)
		to[short] (3+\offset,0)
		;
		\fill (\offset,0) circle (1mm);
		\fill (1.5+\offset,-2.584) circle (1mm);
		\fill (3+\offset,0) circle (1mm);
    \end{circuitikz}
   \caption{Five node network}
  \end{center}
\end{figure}

\begin{Verbatim}[fontsize=\footnotesize]
fiveNodeNetlist
Vt	0 B 1
Rab A B 10k
Cac A C 1e-6
Lbc B C 1e-3
Vcg 0 C 0
.control
	op
	print(v(A))
	.endc
.end
\end{Verbatim}

The first letter of each element line designates the element to simulate: \\
\texttt{V start stop value}$\rightarrow$ Voltage Source between \texttt{start} and \texttt{stop} nodes, \texttt{value} Volts [V]\\
\texttt{R start stop value}$\rightarrow$ Resistor between \texttt{start} and \texttt{stop} nodes, \texttt{value} Ohms [$\Omega$]\\
\texttt{L start stop value}$\rightarrow$ Inductor between \texttt{start} and \texttt{stop} nodes, \texttt{value} Henries [H]\\
\texttt{C start stop value}$\rightarrow$ Capacitor between \texttt{start} and \texttt{stop} nodes, \texttt{value} Farads [F]\\

Node 0 is always designated as ground, and all simulations require a ground node.

The control commands are as follows:\\
\texttt{op} $\rightarrow$ Operating Point Simulation
\texttt{print()} $\rightarrow$ Print the relevant data passed as an argument
\texttt{v(N)} $\rightarrow$ The voltage at node N
\texttt{i(E)} $\rightarrow$ The current through element E

\section{Methods}
The NSA simulator was written in python and uses the methods below.

\subsection{Generate Random Netlist}
\texttt{writeRandomNet(netlist,num,elements):}\\
Generates \texttt{num}-node random graph with no self-linking nodes (symmetric matrix with zeros on the diagonal) for each \texttt{[elements]} type (R,L,C).
Subsequently assigns random values between two realistic limits for each element and writes the network to netlist \texttt{netlist}.
1-1k ohms, 10nF-10uF, 100uH-100mH.

When writing the capacitive and inductive elements, care must be taken to prevent a DC operating point simulation from failing.
The infinite resistance across a capacitor and zero resistance across an inductor are responsible for DC operating point simulation failure, and can be fixed by including a small resistor in series with inductors and a large resistor in parallel with capacitors. 

\begin{tabular}{ c c  p{3cm} }
\texttt{L0 1 2 1mH} & $\rightarrow$ & \texttt{L0 1 tl0 1mH Rl0 tl0 2 1e-5} \\
\end{tabular}
\begin{tabular}{ |c  c  p{3cm} }
\texttt{C0 1 2 1e-6F} & $\rightarrow$ & \texttt{C0 1 2 1e-6F Rc0 1 2 1e8} \\
\end{tabular}

\subsection{Inserting Voltage Sources, Grounds and Probes}
\texttt{insertProbe2(target,nodes,groundNodes,probes,source='DC'):}\\
Inserts a 1V voltage source from ground to each node in \texttt{[nodes]}, grounds each node in \texttt{[groundNodes]}, and adds a voltage print statement for each node in \texttt{[probes]}.
By default, the voltage sources are written as DC sources, but if 'AC' is passed into the last argument the sources are written as AC sources and the AC control statement is added.\\
\texttt{AC dec 5 10m 10meg}\\
Which runs a small-signal AC simulation and returns the amplitudes of the resulting voltage and current waveforms, five sample points per decade from 10mHz to 10MHz.

\subsection{Run Simulation}
\texttt{def runSim(target,results,source='DC'):}\\
Makes a system call to NgSpice in batch mode with netlist \texttt{target} and outputs the result to text file \texttt{results}.
The last argument indicates how to parse the resulting data, as NgSpice returns DC data in the following format:
\begin{Verbatim}[fontsize=\footnotesize]
No. of Data Rows : 1
i(v) = -1.18295e-01
v(5) = 2.532846e-07}
\end{Verbatim}
and AC data is returned in this format:
\begin{Verbatim}[fontsize=\footnotesize]
No. of Data Rows : 6
                                   mynetlist
                                   AC Analysis  Sun Aug 30 18:35:07  2015
--------------------------------------------------------------------------------
Index   frequency       i(v)                            
--------------------------------------------------------------------------------
0	1.000000e+00	-1.72598e+00,	3.978810e+02	
1	1.000000e+01	-1.58689e-01,	3.978827e+01	
2	1.000000e+02	-1.43015e-01,	3.974287e+00	
3	1.000000e+03	-1.42859e-01,	3.520201e-01	
4	1.000000e+04	-1.42857e-01,	-4.18884e-01	
5	1.000000e+05	-1.42857e-01,	-4.58275e+00	

                                   mynetlist
                                   AC Analysis  Sun Aug 30 18:35:07  2015
--------------------------------------------------------------------------------
Index   frequency       v(3)                            
--------------------------------------------------------------------------------
0	1.000000e+00	1.000000e+00,	0.000000e+00	
1	1.000000e+01	1.000000e+00,	0.000000e+00	
2	1.000000e+02	1.000000e+00,	0.000000e+00	
3	1.000000e+03	1.000000e+00,	0.000000e+00	
4	1.000000e+04	1.000000e+00,	0.000000e+00	
5	1.000000e+05	1.000000e+00,	0.000000e+00
\end{Verbatim}
DC data is returned as a dictionary with indices for the voltage source current and each node of interest:\\ \texttt{['i':i(V),'1':v(1),'2':v(2),..]}\\
AC data is returned as a dictionary of lists with indices for the voltage source current and each node of interest:\\ \texttt{['i':[i($V_{f_1}$), i($V_{f_2}$),..., i($V_{f_n}$)], '1':[v($1_{f_1}$), v($1_{f_2}$),..., v($1_{f_n}$)], ...]}

\subsection{Print Matrix}
\texttt{def printMatrix(m):}
Prints matrix \texttt{m} in nice command-line output.
It can handle both 2-dimensional and 3-dimensional matrices, for n-by-n matrices with lists of impedances over many frequencies.

\subsection{Write JSON}
\texttt{def writeJSON(name, network,group, elem):}
Writes a JSON file named \texttt{name.json} of the n-by-n symmetric matrix \texttt{network}.
Three separate JSON files are written, one for each element \texttt{[resistor.json, capacitor.json, and inductor.json]}.
The \texttt{group} parameter is used when drawing multiple schematics on one page.
The \texttt{elem} parameter is used to specify what type of element the network represents.

\section{Executing NSA}
\subsection{Calculate $Z_{n{||}}(f)$}
$Z_{n{||}}(f)$ is found by grounding all nodes except for node n and adding a voltage source to that node, then taking the ratio of the amplitudes of the resulting current into the node of interest and the voltage source.
\subsection{Calculate $V_n(f)$}
$V_n(f)$ is found by grounding all nodes except for nodes n and m, adding a voltage source to node m, and measuring the amplitude of the voltage at node n.
\subsection{Calculate $Z_{nm}(f)$}
$Z_{nm}(f)$ is calculated by the ratio of $Z_{n{||}}(f)$ to $V_n(f)$ scaled by $V_t$.
In the case of this simulation, $V_t$ is one.
\subsection{Finite Difference}
$Z_{nm}$ is passed through the finite-differencer, returning a list $Z'_{nm}[f]$.
$Z'_{nm}[f]$ is thresholded to find the frequencies for which $Z'_{nm}[f]\approx-1$, $Z'_{nm}[f]\approx0$, and $Z'_{nm}[f]\approx1$.
\subsection{Element Identification}
A list of inductances is calculated from $Z_{nm}[f]$ on all frequencies for which $Z'_{nm}[f]\approx-1$, $[f_L]$.\\
$\displaystyle L[f_L] = \frac{|Z_{nm}[f_l]|}{(f_l)}$\\
A list of resistances is calculated from $Z_{nm}[f]$ on all frequencies for which $Z'_{nm}[f]\approx0$, $[f_R]$.\\
$\displaystyle R[f_R] = |Z_{nm}[f_R]|$\\
A list of capacitances is characterized from $Z_{nm}[f]$ on all frequencies for which $Z'_{nm}[f]\approx-1$, $[f_C]$.\\
$\displaystyle C[f_C] = \frac{1}{|Z_{nm}[f_C]|(f_C)}$\\

The element lists are averaged to to filter out inaccurate data.

\subsection{Network Reconstruction}

A new matrix is written for each element type, then the matrices are written to JSON and processed by javascript running D3 to display the final schematic.

\ifdefined\DEBUG
\end{document}
\fi
