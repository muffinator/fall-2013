\documentclass[11pt, a4paper]{article}
\usepackage{tikz}
\usepackage{circuitikz}
\usepackage{float}
\usepackage{amsmath}
\usepackage{graphicx,epsfig}
\usepackage{verbatim}
\usepackage{enumerate}
\usepackage[margin=.8in]{geometry}
\providecommand{\e}[1]{\ensuremath{\times 10^{#1}}}

\title{MEng Thesis \\ Circuit Sensing Breadboard}
\author{Josh Gordonson}

\begin{document}

\section{Network Sensing Algorithm}
To develop a network sensing algorithm, a solution to the following problem must be found:
Given electrical access to an n-node network of linear and time-invariant circuit elements\footnote{ie. resistors, inductors, and capacitors}, determine the elements between any two nodes.
To solve this problem, the networks are broken into small chunks  the set of circuits constructed out of resistors.

\begin{figure}[h]
  \begin{center}
    \begin{circuitikz}
		\draw (0,0)
		node[label={below:$1$}] {}
		to[R=$Z_{13}$] (1.5,2.584)
		node[label={above:$3$}] {}
		to[R=$Z_{35}$] (3,0) % The resistor
		node[label={below:$5$}] {}
		to[R=$Z_{15}$] (0,0)
		to[R=$Z_{12}$] (-1.5,2.584)
		node[label={above:$2$}] {}
		to[R=$Z_{23}$] (1.5,2.584)
		to[R=$Z_{34}$] (4.5,2.584)
		node[label={above:$4$}] {}
		to[R=$Z_{45}$] (3,0);
		\fill (0,0) circle (1mm);
		\fill (3,0) circle (1mm);
		\fill (1.5,2.584) circle (1mm);
		\fill (4.5,2.584) circle (1mm);
		\fill (-1.5,2.584) circle (1mm);
    \end{circuitikz}
   \caption{Example Circuit}
  \end{center}
\end{figure}

\end{document}